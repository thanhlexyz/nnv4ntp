\section{Tutorial}

\begin{frame}{How to create two column slide}
    \begin{columns}
        \begin{column}{0.48\textwidth}
        \end{column}
        \begin{column}{0.48\textwidth}
        \end{column}
    \end{columns}
\end{frame}

\begin{frame}{How to create blocks}
    \mf
    \begin{block}{Normal block}
        Normal block
    \end{block}
    \begin{exampleblock}{Example block}
        Example block
    \end{exampleblock}
    \begin{alertblock}{Alert block}
        Alert block
    \end{alertblock}
\end{frame}

\begin{frame}{How to include figure}
    \begin{figure}
        \centering
        \includegraphics[width=\linewidth]{example-image}
        \caption{Caption}
    \end{figure}
\end{frame}

\begin{frame}{How to include table}
    \mf
    \begin{table}[hp]
        \centering
        \begin{tabular}{@{}ccccc@{}}
            \toprule
            \textbf{Benchmark} & Ma2BEA & MFEA & MaTGA & EBSGA \\ \midrule
            $B_1$    & \textbf{50 (50)} & 0 (0) & 0 (0)           & 0 (0) \\
            $B_2$    & \textbf{50 (50)} & 0 (0) & 0 (0)           & 0 (0) \\
            $B_3$    & 8 (0)            & 0 (0) & \textbf{42 (8)} & 0 (0) \\
            \bottomrule
        \end{tabular}
        \caption{\mf Caption}
        \label{tab:tutorial:example-table}
    \end{table}
\end{frame}

\begin{frame}{How to include pseodo code}
    \mf
    \begin{algorithm}[H]
        \mf
        \caption{\mf Caption}
        \begin{algorithmic}[1]
            \\
            \textbf{Given:} $K$ actions, $T$ rounds.
            \For{round $t \in \{1, \ldots, T\}$}
                \State Choose action $a_t$.
                \State Receive reward $r_t \in [0, 1]$ for the chosen action $a_t$.
            \EndFor
        \end{algorithmic}
        \label{alg:tutorial:example-pseudocode}
    \end{algorithm}
\end{frame}

\begin{frame}{How to fullcite}
    \mf
    \begin{itemize}
        \scriptsize
        \item $\text{[1]}$ \fullcite{mezzavilla2018end}
        \item $\text{[2]}$ \fullcite{larmo2009lte}
    \end{itemize}
\end{frame}

\begin{frame}{How to draw a Tikz diagram}
    \mf
    \begin{figure}[!h]
        \centering
        \begin{tikzpicture}[->, >=stealth', auto, thick, node distance=1cm]
            \tikzstyle{every state}=[fill=white, draw=black, thick, text=black, scale=1]

            \node[state, minimum size=0.5cm] (W1) {$W_{PU}$};
            \node[state, minimum size=0.5cm] (I) at (-135:3) {$I$};
            \node[state, minimum size=0.5cm] (S) at (-45:3) {$S$};
            \node[state, minimum size=0.5cm] (W2) at (-90:3) {$W_{ED}$};

            \path
            (I)  edge node{$\lambda_{PU}$} (W1)
            (I)  edge[below] node{$\lambda_{ED}$} (W2)
            (W1) edge node{$k_{PU}$} (S)
            (W2) edge[below] node{$k_{ED}$} (S)
            (S)  edge[above] node{$\mu$} (I);
        \end{tikzpicture}
        \caption{\mf Semantic-aware CSMA v1, where $k_{PU} = w_{PU}(1-\gamma X_S(t))$, $k_{ED} = w_{ED}(1-\gamma X_S(t))$}
        \label{figure:aoi:semantic-csma-v1}
    \end{figure}
\end{frame}
